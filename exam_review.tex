\documentclass[11pt,a4paper]{article}

% Essential packages
\usepackage[utf8]{inputenc}
\usepackage[margin=1in]{geometry}
\usepackage{amsmath,amssymb,amsthm}
\usepackage{graphicx}
\usepackage{hyperref}
\usepackage{enumitem}
\usepackage{tikz}
\usepackage{tcolorbox}
\usepackage{fancyhdr}
\usepackage{mdframed}
\usepackage{listings}
\usepackage{xcolor}

% Color definitions
\definecolor{theoremblue}{RGB}{0,64,128}
\definecolor{definitiongreen}{RGB}{0,128,0}
\definecolor{exampleorange}{RGB}{255,140,0}
\definecolor{notegray}{RGB}{128,128,128}
\definecolor{highlightyellow}{RGB}{255,255,200}

% Theorem environments
\newtheorem{theorem}{Theorem}[section]
\newtheorem{lemma}[theorem]{Lemma}
\newtheorem{proposition}[theorem]{Proposition}
\newtheorem{corollary}[theorem]{Corollary}
\newtheorem{definition}[theorem]{Definition}
\newtheorem{example}[theorem]{Example}
\newtheorem{remark}[theorem]{Remark}

% Custom environments for review notes
\newtcolorbox{reviewbox}[1][]{
    colback=blue!5!white,
    colframe=blue!75!black,
    title=#1,
    fonttitle=\bfseries
}

\newtcolorbox{keypoint}{
    colback=yellow!10!white,
    colframe=orange!75!black,
    title=Key Point,
    fonttitle=\bfseries
}

\newtcolorbox{formula}{
    colback=green!5!white,
    colframe=green!75!black,
    title=Important Formula,
    fonttitle=\bfseries
}

\newtcolorbox{warning}{
    colback=red!5!white,
    colframe=red!75!black,
    title=Warning,
    fonttitle=\bfseries
}

% Header and footer
\pagestyle{fancy}
\fancyhf{}
\rhead{\thepage}
\lhead{Notes Review: \coursename}
\cfoot{\small Last updated: \today}

% Custom commands
\newcommand{\coursename}{Course Name Here}
\newcommand{\topicname}{Topic Name Here}
\newcommand{\highlight}[1]{\colorbox{highlightyellow}{#1}}
\newcommand{\important}[1]{\textbf{\textcolor{red}{#1}}}
\newcommand{\concept}[1]{\textbf{\textcolor{theoremblue}{#1}}}
\newcommand{\R}{\mathbb{R}}
\newcommand{\N}{\mathbb{N}}
\newcommand{\Z}{\mathbb{Z}}
\newcommand{\C}{\mathbb{C}}

% Code listing settings
\lstset{
    basicstyle=\ttfamily\small,
    breaklines=true,
    commentstyle=\color{gray},
    keywordstyle=\color{blue},
    numberstyle=\tiny\color{gray},
    stringstyle=\color{red},
    showstringspaces=false,
    captionpos=b
}

\begin{document}

% Title page
\begin{titlepage}
    \centering
    \vspace*{2cm}
    {\Huge \textbf{Notes Review}\\[0.5cm]}
    {\LARGE \coursename\\[0.3cm]}
    {\Large \topicname\\[1cm]}
    {\large Your Name\\[0.5cm]}
    {\large \today\\[2cm]}
    
    \begin{abstract}
        This document contains a comprehensive review of notes for [Course/Topic]. 
        It includes key concepts, formulas, examples, and practice problems.
    \end{abstract}
\end{titlepage}

% Table of contents
\tableofcontents
\newpage

% Section 1: Overview
\section{Overview}

\begin{reviewbox}[Topic Summary]
    Provide a brief overview of the main topic and its importance.
    \begin{itemize}
        \item Main concept 1
        \item Main concept 2
        \item Main concept 3
    \end{itemize}
\end{reviewbox}

\section{TODO}

\begin{reviewbox}[Topic Summary]
    TODO exam problems
    \begin{itemize}
        \item SS 2023 2c. sparse grids
        \item SS 2023 3c. space filling curves
    \end{itemize}
\end{reviewbox}

% Section 2: Key Definitions
\section{Key Definitions}

\begin{definition}[Important Term]
    Define the important term here. For example:
    
    A \concept{function} $f: A \to B$ is a relation that assigns to each element 
    $a \in A$ exactly one element $b \in B$.
\end{definition}

\begin{definition}[Understand how to convert between trigonometric and exponential using Euler's formula]
    Define the important term here. For example:
    
    A \concept{function} $f: A \to B$ is a relation that assigns to each element 
    $a \in A$ exactly one element $b \in B$.
\end{definition}


\begin{keypoint}
    Remember that definitions are the foundation of understanding. 
    Make sure you can state them precisely!
\end{keypoint}

% Section 3: Main Theorems
\section{Main Theorems and Results}

\begin{theorem}[Fundamental Theorem]
    State the theorem here. For example:
    
    Let $f: [a,b] \to \R$ be continuous. Then $f$ attains its maximum and minimum 
    values on $[a,b]$.
\end{theorem}

\begin{proof}
    Sketch of proof or key ideas...
\end{proof}

\begin{formula}
    Key formula to remember:
    \begin{equation}
        \int_a^b f(x)\,dx = F(b) - F(a)
    \end{equation}
    where $F'(x) = f(x)$.
\end{formula}

% Section 4: Examples
\section{Examples and Applications}

\begin{example}[Worked Example]
    Consider the function $f(x) = x^2$. We want to find...
    
    \textbf{Solution:}
    \begin{align}
        f'(x) &= 2x \\
        f''(x) &= 2
    \end{align}
    
    Therefore, the function has a minimum at $x = 0$.
\end{example}

\begin{warning}
    Common mistake: Don't forget to check the domain when finding extrema!
\end{warning}

\section{Endterm Summer 2025}

\begin{example}[Problem 1a: Quarter-Wave Fourier coefficients are purely imaginary]
Given a real-valued input dataset $f_0, \ldots, f_{2N-1}$ with the symmetry condition $f_{2N-n-1} = -f_n$, show that the Quarter-Wave Fourier coefficients
$$F_k = \frac{1}{2N} \sum_{n=0}^{2N-1} f_n \omega_{2N}^{-k(n+\frac{1}{2})}$$
have only imaginary values and can be written as
$$F_k = -\frac{i}{N} \sum_{n=0}^{N-1} f_n \sin\left(\frac{\pi k}{N}\left(n + \frac{1}{2}\right)\right)$$

\textbf{Solution:}

\begin{enumerate}
\item \textbf{Split the sum:} Separate the sum into two parts
$$F_k = \frac{1}{2N}\left(\sum_{n=0}^{N-1} f_n \omega_{2N}^{-k(n+\frac{1}{2})} + \sum_{n=N}^{2N-1} f_n \omega_{2N}^{-k(n+\frac{1}{2})}\right)$$

\item \textbf{Apply symmetry condition:} Use $f_n = -f_{2N-n-1}$ in the second sum and change variables

\item \textbf{Combine terms:} After simplification, obtain
$$F_k = \frac{1}{2N} \sum_{n=0}^{N-1} f_n \left(\omega_{2N}^{-k(n+\frac{1}{2})} - \omega_{2N}^{k(n+\frac{1}{2})}\right)$$

\item \textbf{Use Euler's formula:} Since $e^{-ix} - e^{ix} = -2i\sin(x)$, we get
$$F_k = -\frac{i}{N} \sum_{n=0}^{N-1} f_n \sin\left(\frac{\pi k}{N}\left(n + \frac{1}{2}\right)\right)$$
\end{enumerate}

\textbf{Key insight:} The anti-symmetry condition $f_{2N-n-1} = -f_n$ causes the real parts to cancel, leaving only imaginary sine components.
\end{example}

\begin{example}[Problem 1b: Symmetry condition for QW-DST coefficients]
Show that the coefficients $F_k$ of the QW-DST satisfy the symmetry condition $F_k = F_{2N-k}$.

\textbf{Solution:} 

\textbf{Starting formula:} From part (a), we have
$$F_k = -\frac{i}{N} \sum_{n=0}^{N-1} f_n \sin\left(\frac{\pi k}{N}\left(n + \frac{1}{2}\right)\right)$$

\textbf{Compute $F_{2N-k}$:}
$$F_{2N-k} = -\frac{i}{N} \sum_{n=0}^{N-1} f_n \sin\left(\frac{\pi (2N-k)}{N}\left(n + \frac{1}{2}\right)\right)$$

\textbf{Simplify the sine argument:}
$$\frac{\pi (2N-k)}{N}\left(n + \frac{1}{2}\right) = 2\pi n + \pi - \frac{\pi k}{N}\left(n + \frac{1}{2}\right)$$

\textbf{Apply sine properties:}
\begin{itemize}
\item $\sin(2\pi n + x) = \sin(x)$ (periodicity)
\item $\sin(\pi - x) = \sin(x)$ (symmetry)
\end{itemize}

Therefore: $F_{2N-k} = F_k$

\textbf{Key insight:} Due to this symmetry, we only need to compute $F_k$ for $k = 0, \ldots, N$ rather than all $2N$ coefficients, reducing computational requirements by half.
\end{example}

\begin{example}[Problem 1c: Computing coefficients for QW-DST using real FFT] 
We assume a procedure \texttt{real-FFT(g,N)} that computes Fourier coefficients $G_k$ efficiently on real dataset $g$ that consists of $2N$ values $g_n$. Use this procedure to compute coefficients $F_k$ for $k = 0,\ldots,N-1$ from equation (2) for (non-symmetrical) real data $f_0,\ldots,f_{N-1}$ stored in parameter field $g$.

  \textbf{Solution:} Pre-process the data using anti-symmetric extension, perform real-FFT, then post-process coefficients using the relationship between DFT and QW-DST coefficients.
  
  \begin{enumerate}
  \item \textbf{Create anti-symmetric data sequence of length $2N$:}
  \begin{align}
  g_n &= f_n, \quad \text{for } n = 0,1,\ldots,N-1\\
  g_{2N-n-1} &= -f_n, \quad \text{for } n = 0,1,\ldots,N-1
  \end{align}
  
  \item \textbf{Run real-FFT to compute $G_k$ from $g_n$:}
  $$G_k = \texttt{real-FFT}(g,N), \quad \text{for } k = 0,\ldots,N$$
  
  \item \textbf{Extract QW-DST coefficients from FFT coefficients:}
  $$F_k = G_k \omega_{2N}^{-k/2} = G_k e^{-i\pi k/(2N)}, \quad \text{for } k = 0,\ldots,N-1$$
  \end{enumerate}

  \textbf{Key insight:} The QW-DST can be computed from a standard FFT by exploiting the anti-symmetric extension and the phase shift relationship $F_k = G_k \omega_{2N}^{-k/2}$ between the transforms.
\end{example}


\begin{example}[Problem 2b: Admissibility condition for wavelets]
   Given the drawing from part (a), what conclusions can you make about the integral of the obtained wavelet? What does this imply for wavelets in general?
   
   \textbf{Solution:}
   
   \textbf{Observation from the graph:} The area above the x-axis equals the area below the x-axis. Therefore:
   $$\int_{-\infty}^{+\infty} \psi_1(t)\,dt = 0$$
   
   \textbf{Implications for wavelets in general:}
   \begin{enumerate}
       \item \textbf{Admissibility condition:} All wavelets must satisfy $\int_{-\infty}^{+\infty} \psi(t)\,dt = 0$
       
       \item \textbf{Zero-mean property:} Wavelets oscillate around zero and represent fluctuations relative to the average
       
       \item \textbf{Complementary roles:}
       \begin{itemize}
           \item Scaling functions $\phi(t)$ capture average (DC) information
           \item Wavelets $\psi(t)$ capture details and variations
       \end{itemize}
       
       \item \textbf{Signal processing interpretation:} Wavelets act as band-pass filters that:
       \begin{itemize}
           \item Cannot detect constant signals
           \item Capture oscillations and local changes
           \item Are ideal for multi-scale analysis while preserving edges and transitions
       \end{itemize}
   \end{enumerate}
   
   \textbf{Key insight:} The zero-integral property ensures wavelets are orthogonal to constant functions, making them perfect for decomposing signals into different frequency bands and scales.
\end{example}

\begin{example}[Problem 2b: Admissibility condition for wavelets]
    Given the drawing from part (a), what conclusions can you make about the integral of the obtained wavelet. What does this imply for wavelts in general?
    
    \textbf{Solution:}
    \begin{align}
        \text{1. Area above x-axis equals area below x-axis } This means the integral of the wavelt function is 0. \\
        \text{2. This implies the mean is zero.} The wavelet basis functions will represent fluctuations to the average of the coarser level. \\
        \text{3. The admissibility condition is met } The integral of the wavelet over all values is zero. \\
    \end{align}
    Main idea: Scaling function captures avergae (DC) information, while wavelts capture details. Wavelets act as band-pass filters and cannot detect constant siganls. They capture oscillations and local changes on signals.
    Intuition: Zero-mean prop makes wavelets ideal for analyzing signals at different scales while preserving important features like edges and transitions.

\end{example}

\begin{example}[Problem 3b: Hierarchical surpluses for smooth functions]
   Consider hierarchical interpolant on $(N+1) \times (N+1)$ grids for increasing number of grid points $N = 2^L$. Characterize how the size of surpluses decreases for sufficiently smooth functions $f$.
   
   \textbf{Solution:}
   
   \textbf{1. Along horizontal/vertical grid lines:} Hierarchical surpluses decrease by $\mathbf{1/4}$ from level to level.
   \begin{itemize}
       \item These represent 1D refinement where grid spacing halves: $h \to h/2$
       \item Surpluses are proportional to $h^2 \cdot |f''|$
       \item Since $(h/2)^2 = h^2/4$, surpluses decrease by factor of $1/4$
   \end{itemize}
   
   \textbf{2. Along diagonal grid lines:} Hierarchical surpluses decrease by $\mathbf{1/16}$.
   \begin{itemize}
       \item Points added at diagonal positions constitute 2D refinement
       \item Both $x$ and $y$ directions refined simultaneously
       \item Surplus proportional to $h_x^2 \cdot h_y^2$
       \item When $h_x \to h_x/2$ and $h_y \to h_y/2$: $(h_x/2)^2 \cdot (h_y/2)^2 = h_x^2 h_y^2/16$
   \end{itemize}
   
   \textbf{3. Absolute size depends on derivatives:}
   \begin{itemize}
       \item Horizontal/vertical points: proportional to pure second derivatives $\frac{\partial^2 f}{\partial x^2}$, $\frac{\partial^2 f}{\partial y^2}$
       \item Diagonal points: proportional to mixed second derivative $\frac{\partial^2 f}{\partial x \partial y}$
   \end{itemize}
   
   \textbf{Key insight:} The rapid decrease of surpluses (especially $1/16$ for diagonal points) explains why sparse grids can efficiently approximate smooth functions in high dimensions by omitting many grid points while maintaining accuracy.
\end{example}

\begin{example}[Problem 4a: Sparse grids data structures]
    Name the different data structures used for sparse grids and discuss the following points: hierarchization/dehierarchization (consider data access and traversal complexity), spatial adaptivity, and memory consumption. 
   
   \textbf{Solution:}
   
   \textbf{1. Arrays} Grid points are stored in a contiguous array. A mapping is needed from hierarchical index (l,i) and flat index j.
   \begin{itemize}
       \item Hierarchization/dehierarchization: access by index O(1), but need mapping between hierarchical index (l,i) and flat 1D index. Hierarchical neighbors can be deduced directly from current node's hierarchical index.
       \item Spatial adaptivity: cannot add or delete elements; does not support spatial adaptivity. However, dimensional adaptivity can be accomodated.
       \item Memory consumption: only data is tored i.e., v_j at each grid point. No need to store (l,i) if correct mapping is provided. Cache efficient!
   \end{itemize}
   
   \textbf{2. Along diagonal grid lines:} Hierarchical surpluses decrease by $\mathbf{1/16}$.
   \begin{itemize}
       \item Points added at diagonal positions constitute 2D refinement
       \item Both $x$ and $y$ directions refined simultaneously
       \item Surplus proportional to $h_x^2 \cdot h_y^2$
       \item When $h_x \to h_x/2$ and $h_y \to h_y/2$: $(h_x/2)^2 \cdot (h_y/2)^2 = h_x^2 h_y^2/16$
   \end{itemize}
   
   \textbf{3. Absolute size depends on derivatives:}
   \begin{itemize}
       \item Horizontal/vertical points: proportional to pure second derivatives $\frac{\partial^2 f}{\partial x^2}$, $\frac{\partial^2 f}{\partial y^2}$
       \item Diagonal points: proportional to mixed second derivative $\frac{\partial^2 f}{\partial x \partial y}$
   \end{itemize}
   
   \textbf{Key insight:} The rapid decrease of surpluses (especially $1/16$ for diagonal points) explains why sparse grids can efficiently approximate smooth functions in high dimensions by omitting many grid points while maintaining accuracy.
\end{example}

\begin{example}[Problem 5b: Arithmetic representation of the Peano-Meander Curve]
    You are given the parametrization q(t) of the Peano-Meander curve, where t is the representation t in base nine system. Determine the operators Q1,Q4,Q6.
   
   \textbf{Solution:}
   
   \textbf{1. Layout the curve} Divide the curve into blocks Q0-Q8. 
   
   \textbf{2. Along diagonal grid lines:} Identify different patterns and whether reflection/coordinate swap is needed.
   
   \textbf{3. Absolute size depends on derivatives:}

   
   \textbf{Key insight:} The rapid decrease of surpluses (especially $1/16$ for diagonal points) explains why sparse grids can efficiently approximate smooth functions in high dimensions by omitting many grid points while maintaining accuracy.
\end{example}

\begin{example}[Problem 5b 2: Computing coordinates on Peano-Meander curve]
    You are given the parametrization q(t) of the Peano-Meander curve. You are given coordinates of q(2/3) and q(1/2).
   
   \textbf{Solution:}
   
   \textbf{1. Determine nonary representation} For q(2/3), 2/3 = 0_9.6. q(2/3) = Q6(0 0) = (2/3 2/3). For q(1/2), 1/2 = 0_9.44444...
   
   \textbf{2. Write the representation in operator form} q(1/2) = Q4 * Q4 * ... (0 0) = lim n->inf Q_4^n (0 0)
   
   \textbf{3. Absolute size depends on derivatives:}

   
   \textbf{Key insight:} The rapid decrease of surpluses (especially $1/16$ for diagonal points) explains why sparse grids can efficiently approximate smooth functions in high dimensions by omitting many grid points while maintaining accuracy.
\end{example}

\section{Endterm Summer 2023}

\begin{example}[Problem 1: Discrete Fourier Periodicity]
    Derive the relationship between $F_{k}$ and $G_{2k}$ for an N-periodic dataset f.
    
    \textbf{Solution:}
    \begin{align}
        \text{1. Write the definition of } G_{2k}: \quad & G_{2k} = \frac{1}{2N}\sum_{n=0}^{2N-1} f_n e^{-i2\pi n(2k)/(2N)} \\
        \text{2. Simplify the exponential: } & e^{-i2\pi n(2k)/(2N)} = e^{-i2\pi nk/N} \\
        \text{3. Split the sum into 2 parts: } & \sum_{n=0}^{2N-1} = \sum_{n=0}^{N-1} + \sum_{n=N}^{2N-1} \\
        \text{4. Use periodicity: substitute } & n = N + j \text{ where } f_{N+j} = f_j \\
        \text{5. Combine the sums} & \text{(both terms become identical)} \\
        \text{6. Result: } & G_{2k} = \frac{1}{2N} \cdot 2 \sum_{n=0}^{N-1} f_n e^{-i2\pi nk/N} = F_k
    \end{align}
    Main idea: When applying a 2N-point DFT to N-periodic data, the even-indexed coefficients ($G_0$, $G_2$, $G_4$, , $G_{2N-2}$) are identical to the coefficients of an N-point DFT ($F_0$, $F_1$, $F_2$, $F_{N-1}$).
    Intuition: This happens because the N-periodic data repeats itself in the second half of the 2N samples. Each data value appears exactly twice in the 2N-point transform, effectively doubling its contribution, which when normalized by the 1/(2N) factor, gives the same result as the N-point DFT.

\end{example}

\begin{example}[Problem 1b: Discrete Sine Transform]
    Given a $2N$ data set $f_{-N+1}, \ldots, f_0, f_1, \ldots, f_N$ with antisymmetry condition $f_{-n} = -f_n$ and all $f_n \in \mathbb{R}$, prove that:
    $$F_k = \frac{1}{2N} \sum_{n=-N+1}^{N} f_n \omega_{2N}^{nk} = \frac{-i}{N} \sum_{n=1}^{N-1} f_n \sin\left(\frac{\pi nk}{N}\right)$$
    
    \textbf{Solution:}
    
    \textbf{Step 1:} Apply antisymmetry to find zero values
    \begin{itemize}
        \item At $n=0$: $f_0 = -f_{-0} = -f_0 \Rightarrow f_0 = 0$
        \item At $n=N$: By consistency with antisymmetry, $f_N = 0$
    \end{itemize}
    
    \textbf{Step 2:} Expand the DFT sum and eliminate zero terms
    $$F_k = \frac{1}{2N} \left[\sum_{n=-N+1}^{-1} f_n \omega_{2N}^{nk} + \sum_{n=1}^{N-1} f_n \omega_{2N}^{nk}\right]$$
    
    \textbf{Step 3:} Apply antisymmetry to the negative index sum
    \begin{itemize}
        \item Substitute $m = -n$ in the first sum
        \item Use $f_{-m} = -f_m$ to get:
    \end{itemize}
    $$\sum_{n=-N+1}^{-1} f_n \omega_{2N}^{nk} = -\sum_{m=1}^{N-1} f_m \omega_{2N}^{-mk}$$
    
    \textbf{Step 4:} Combine the sums
    $$F_k = \frac{1}{2N} \sum_{n=1}^{N-1} f_n \left[\omega_{2N}^{nk} - \omega_{2N}^{-nk}\right]$$
    
    \textbf{Step 5:} Express $\omega_{2N}$ and apply Euler's formula
    \begin{itemize}
        \item $\omega_{2N} = e^{2\pi i/2N} = e^{\pi i/N}$
        \item $\omega_{2N}^{nk} - \omega_{2N}^{-nk} = e^{\pi ink/N} - e^{-\pi ink/N} = 2i\sin\left(\frac{\pi nk}{N}\right)$
    \end{itemize}
    
    \textbf{Step 6:} Substitute and simplify
    $$F_k = \frac{1}{2N} \sum_{n=1}^{N-1} f_n \cdot 2i\sin\left(\frac{\pi nk}{N}\right) = \frac{i}{N} \sum_{n=1}^{N-1} f_n \sin\left(\frac{\pi nk}{N}\right)$$
    
    \textbf{Note:} The negative sign in the final result depends on the DFT sign convention used.
    
    \textbf{Main idea:} For antisymmetric real data, the DFT reduces to a purely imaginary Discrete Sine Transform, containing only sine terms due to the odd symmetry of the input.
\end{example}

\begin{example}[Problem 1c: Computing DST coefficients using FFT]
    Given a dataset $f_n$ for $n = 1, \ldots, N-1$, use the result from part (b) to efficiently compute the Discrete Sine Transform (DST) coefficients using FFT.
    
    \textbf{Solution:}
    
    \textbf{Step 1: Preprocessing - Create antisymmetric extension}
    \begin{itemize}
        \item Given: size-$N$ vector with values $f_1, f_2, \ldots, f_{N-1}$
        \item Create size-$2N$ antisymmetric vector:
        \begin{align}
            f_0 &= 0\\
            f_n &= f_n \text{ for } n = 1, \ldots, N-1\\
            f_N &= 0\\
            f_{-n} &= -f_n \text{ for } n = 1, \ldots, N-1
        \end{align}
    \end{itemize}
    
    \textbf{Step 2: Apply FFT}
    Compute $F_k$ using standard FFT on the $2N$-point antisymmetric sequence.
    
    \textbf{Step 3: Postprocessing - Extract real DST coefficients}
    From part (b), we know that for antisymmetric data:
    $$F_k = \frac{-i}{N} \sum_{n=1}^{N-1} f_n \sin\left(\frac{\pi nk}{N}\right)$$
    
    To obtain the real-valued DST coefficients:
    $$\hat{F}_k = -\text{Im}\{F_k\} = \frac{1}{N} \sum_{n=1}^{N-1} f_n \sin\left(\frac{\pi nk}{N}\right)$$
    
    for $k = 1, \ldots, N-1$.
    
    \textbf{Key insight:} The antisymmetric extension causes the DFT to produce purely imaginary coefficients proportional to the sine series. By extracting the imaginary part and flipping the sign, we obtain the real-valued DST coefficients efficiently using FFT.
\end{example}

\begin{example}[Problem 2: Sparse Grids]
    The goal is to represent a 2D function using a sparse grid of level 3. First, specify which basis functions are used by giving exact ranges for level indices $\vec{l}$ and $\vec{i}$. For each basis function used on the sparse grid, mark the corresponding point in the tableau of subspaces shown in Figure 2.1.
    
    \textbf{Solution:}
    \begin{enumerate}
        \item \textbf{Identify subspaces to include:} For sparse grid $S_3$, use subspaces where $|\vec{l}|_1 = l_1 + l_2 \leq 3 + 1 = 4$.\\
        This gives the combinations: $(1,1)$, $(1,2)$, $(1,3)$, $(2,1)$, $(2,2)$, $(3,1)$.
        
        \item \textbf{Determine basis functions for each subspace:} \\
        Index set: $I_{\vec{l}} = \{\vec{i} : 1 \leq i_d < 2^{l_d}, \text{ all } i_d \text{ odd}\}$\\
        We only use odd indices to avoid redundancy with coarser levels.
        
        \item \textbf{Mark grid points:} Each basis function $\phi_{\vec{l},\vec{i}}$ corresponds to a grid point $\vec{x}_{\vec{l},\vec{i}} = (i_1 \cdot 2^{-l_1}, i_2 \cdot 2^{-l_2})$ in the tableau.
    \end{enumerate}
    
    \textbf{Main idea:} Sparse grids efficiently approximate multi-dimensional functions without succumbing to the curse of dimensionality. They achieve this by:
    \begin{itemize}
        \item Using a hierarchical structure that builds on coarser grids
        \item Selecting only odd-indexed basis functions to avoid redundancy
        \item Balancing computational efficiency with approximation accuracy
    \end{itemize}
    Each grid point corresponds to one basis function in the sparse grid representation.
\end{example}


\begin{example}[Problem 2b: Coefficients for sparse grid approximation]
    Identify which coefficients must be computed to evaluate a 2D function at the point $(1/3, 2/3)$ using sparse grid approximation.
    
    \textbf{Solution:}
    \begin{enumerate}
        \item \textbf{Identify relevant subspaces:} For sparse grid level $n = 3$ in dimension $d = 2$, the subspaces satisfying $|l|_1 \leq n + d - 1 = 4$ are:
        \begin{itemize}
            \item $(l_1, l_2) \in \{(1,1), (1,2), (1,3), (2,1), (2,2), (3,1)\}$
        \end{itemize}
        
        \item \textbf{Define index sets $I_l$:} For each level $l$, the index set contains odd integers:
        \begin{itemize}
            \item $I_1 = \{1\}$
            \item $I_2 = \{1, 3\}$
            \item $I_3 = \{1, 3, 5, 7\}$
        \end{itemize}
        
        \item \textbf{Identify grid points:} For each subspace $(l_1, l_2)$ and indices $(i_1, i_2) \in I_{l_1} \times I_{l_2}$, compute:
        $$x_{l,i} = \left(\frac{i_1}{2^{l_1}}, \frac{i_2}{2^{l_2}}\right)$$
        
        \item \textbf{Select relevant basis functions:} The hierarchical basis function $\phi_{l,i}(x)$ centered at $x_{l,i}$ has support on:
        $$\left[\frac{i_1-1}{2^{l_1}}, \frac{i_1+1}{2^{l_1}}\right] \times \left[\frac{i_2-1}{2^{l_2}}, \frac{i_2+1}{2^{l_2}}\right]$$
        
        Include coefficient $a_{l,i}$ if and only if $(1/3, 2/3)$ lies within this support.
    \end{enumerate}
    
    \textbf{Main idea:} Sparse grids efficiently approximate multi-dimensional functions by:
    \begin{itemize}
        \item Using hierarchical basis functions with local support
        \item Including only basis functions whose support contains the evaluation point
        \item Exploiting the tensor product structure while avoiding the curse of dimensionality
    \end{itemize}
    
    The coefficients to evaluate correspond to basis functions that are non-zero at $(1/3, 2/3)$.

    \textbf{Local support} refers to the region where a basis function is non-zero. Outside this region, the function equals zero.

    \textbf{Hierarchical basis functions} in sparse grids are hat functions (piecewise linear) that resemble triangular tents. To evaluate $f$ at a given point, we only need coefficients for basis functions whose support contains that point.

\end{example}

\begin{example}[Problem 3: Space-filling curves]

The goal is to derive the operators $M_0$ through $M_8$ that transform the first iteration of the Meurthe (Peano) curve into each sub-square of the second iteration.

\textbf{Solution Approach:}

For each operator $M_i$, determine the transformation needed by analyzing:

\begin{enumerate}
    \item \textbf{Scaling:} Since we divide into a $3 \times 3$ grid, scale by $\frac{1}{3}$ in both directions.
    
    \item \textbf{Rotation/Reflection:} Determine what rotation or reflection is needed so the curve segment connects properly with adjacent sub-squares.
    
    \item \textbf{Translation:} Determine the offset to position the transformed curve in the correct sub-square.
\end{enumerate}

\textbf{Method:} For each sub-square $i$:
\begin{itemize}
    \item Identify how the curve should enter and exit the sub-square
    \item Determine what transformation of the original square pattern achieves this
    \item Express as $M_i(x,y) = (ax + by + c, dx + ey + f)$
\end{itemize}

\textbf{Example for $M_0$:}
\begin{itemize}
    \item Sub-square 0 is at position $(0,0)$ to $(\frac{1}{3}, \frac{1}{3})$
    \item The transformation $M_0(x,y) = (\frac{y}{3}, \frac{x}{3})$ swaps $x$ and $y$ (90° rotation) and scales by $\frac{1}{3}$
    \item This creates the correct curve segment that starts at $(0,0)$ and connects to sub-square 1
\end{itemize}

\end{example}

\begin{example}[Problem 3b: Space-filling curves]

Find the parameter $t$ such that $m(t) = (1/2, 1/2)$.

\textbf{Solution:}

\begin{enumerate}
    \item \textbf{Locate $(1/2, 1/2)$ in the grid:} 
    In the second iteration's $3 \times 3$ grid, the point $(1/2, 1/2)$ lies in the center sub-square (sub-square 4). Therefore, the first digit is $n_1 = 4$, so $t$ starts with $0_9.4\ldots$
    
    \item \textbf{Check if $(1/2, 1/2)$ is a fixed point:}
    Given $M_4(x,y) = (-y/3 + 2/3, -x/3 + 2/3)$, we solve:
    \begin{align}
        1/2 &= -y/3 + 2/3 \Rightarrow y = 1/2\\
        1/2 &= -x/3 + 2/3 \Rightarrow x = 1/2
    \end{align}
    Therefore $M_4(1/2, 1/2) = (1/2, 1/2)$, confirming $(1/2, 1/2)$ is a fixed point of $M_4$.
    
    \item \textbf{Determine subsequent digits:}
    Since $(1/2, 1/2)$ is a fixed point of $M_4$, it remains in sub-square 4 at every level. Therefore, all digits are 4: $t = 0_9.\overline{4}$
    
    \item \textbf{Convert to decimal:}
    \begin{align}
        0_9.\overline{4} &= \frac{4}{9} + \frac{4}{81} + \frac{4}{729} + \ldots \\
        &= \frac{4}{9} \cdot \frac{1}{1-\frac{1}{9}} = \frac{4}{9} \cdot \frac{9}{8} = \frac{1}{2}
    \end{align}
    
    \item \textbf{Verification:}
    Since $t = 1/2 = 0_9.\overline{4}$:
    $$m(1/2) = M_4 \circ M_4 \circ M_4 \circ \cdots (0,0) = \lim_{n \to \infty} M_4^n(0,0) = (1/2, 1/2)$$
    The limit converges to the fixed point of $M_4$.
\end{enumerate}

\textbf{Key insights:}
\begin{itemize}
    \item The center of parameter space ($t = 1/2$) maps to the center of the unit square $((1/2, 1/2))$.
    \item Fixed point analysis simplifies the recursive calculation.
    \item Function composition ($M_4 \circ M_4 \circ \cdots$) determines the mapping, while addition converts the base-9 representation to decimal.
    \item The expression $M_4 \circ M_4 \circ M_4 \circ \cdots (0,0)$ is the expansion of the Meurthe function $m$ applied to $t = 0_9.\overline{4}$ (which equals $1/2$). For infinite compositions to converge, the limit point must be a fixed point of the operator—a fundamental property of space-filling curves.
\end{itemize}

\textbf{Answer:} $t = 1/2$
\end{example}

\begin{example}[Problem 4: Haar Wavelet Transform in 1D]
The goal is to implement a Python function that performs the Haar wavelet transform on a 1D data sequence. The Haar wavelet is used in multi-resolution data analysis, where each element in the transformed array provides information about the data at different scales.
\textbf{Solution Approach:}
The Haar wavelet transform recursively decomposes a signal into averages (low-frequency components) and differences (high-frequency details). Given a filter function that processes pairs of elements, we apply it recursively to achieve a complete multi-resolution decomposition.
\textbf{Implementation:}
\begin{lstlisting}[language=Python]
def dwt_1d(c, jmax):
"""
Performs 1D Discrete Wavelet Transform using Haar wavelets.
Parameters:
- c: Input array (modified in-place)
- jmax: Current size of the array to process

After transformation:
- c[0]: Overall average of the entire signal
- c[1]: Low-frequency detail (difference between halves)
- c[2:]: Higher-frequency details at successive scales
"""
# Base case: nothing to do for single element
if jmax == 1:
    return

# Apply filter to current level
# This computes averages in first half, differences in second half
filter_1d(c, jmax)

# Recursively process the averages (first half)
# This creates a multi-resolution representation
dwt_1d(c, jmax // 2)
\end{lstlisting}
\textbf{How it works:}
\begin{enumerate}
\item \textbf{Level 1:} Process all elements, storing averages in the first half and differences in the second half
\item \textbf{Level 2:} Process only the averages from Level 1, further decomposing them
\item \textbf{Continue recursively} until only one average remains
\end{enumerate}
\textbf{Example:} For array [2, 4, 6, 8]:
\begin{itemize}
\item Initial: [2, 4, 6, 8]
\item After Level 1: [3, 7, -1, -1] (averages: 3, 7; differences: -1, -1)
\item After Level 2: [5, -2, -1, -1] (overall average: 5; scale differences: -2, -1, -1)
\end{itemize}
The final array contains the signal decomposed into different frequency components, useful for compression, denoising, and analysis.
\end{example}

\begin{example}[Problem 4b: Haar Wavelet Transform in 2D]
Implement a Python function that performs the 2D Haar wavelet transform using the provided 1D implementation.

\textbf{Solution Approach:}
The 2D Haar wavelet transform applies the 1D transform along rows and columns, creating a multi-resolution decomposition with four quadrants at each level.

\textbf{Implementation:}
\begin{lstlisting}[language=Python]
def dwt_2d(image, jmax):
    """
    Performs 2D Discrete Wavelet Transform using Haar wavelets.
    
    Parameters:
    - image: 2D array of size 2^p x 2^p (modified in-place)
    - jmax: Current size of the subimage to process
    
    After transformation, the image contains:
    - Top-left (LL): Low frequencies in both directions
    - Top-right (LH): Vertical edges (horizontal high-freq)
    - Bottom-left (HL): Horizontal edges (vertical high-freq)  
    - Bottom-right (HH): Diagonal edges and noise
    """
    # Base case
    if jmax <= 1:
        return
    
    # Apply 1D DWT along all rows
    for row in range(jmax):
        row_data = image_row(image, row, jmax)
        dwt_1d(row_data, jmax)
        for col in range(jmax):
            image[row][col] = row_data[col]
    
    # Apply 1D DWT along all columns
    for col in range(jmax):
        column_data = image_col(image, col, jmax)
        dwt_1d(column_data, jmax)
        for row in range(jmax):
            image[row][col] = column_data[row]
    
    # Recursive call on top-left quadrant (LL)
    dwt_2d(image, jmax // 2)
\end{lstlisting}

\textbf{How it works:}
\begin{enumerate}
    \item \textbf{Row transformation:} Apply 1D DWT to each row, splitting the image into left (low-frequency) and right (high-frequency) halves.
    
    \item \textbf{Column transformation:} Apply 1D DWT to each column, creating four quadrants:
    \begin{itemize}
        \item \textbf{LL (top-left):} Averages of averages - smoothed version
        \item \textbf{LH (top-right):} Row averages, column differences - vertical edges
        \item \textbf{HL (bottom-left):} Row differences, column averages - horizontal edges
        \item \textbf{HH (bottom-right):} Differences of differences - diagonal edges and noise
    \end{itemize}
    
    \item \textbf{Recursive decomposition:} Process the LL quadrant recursively to create a multi-resolution representation.
\end{enumerate}

\textbf{Key insight:} Averages act as low-pass filters (preserving smooth variations), while differences act as high-pass filters (capturing rapid changes and edges).

\textbf{Further questions:} Does it matter if you process rows first or columns first?
\end{example}

\begin{example}[Problem 4c: Deriving formula for inverse Haar wavelet coefficients]
We are given the formulas which describe the averaging and differencing performed by the forward Haar wavelet transform:
\begin{align}
c_j^{(l)} &= \frac{1}{2}c_{2j}^{(l+1)} + \frac{1}{2}c_{2j+1}^{(l+1)} \quad \text{(averaging)}\\
d_j^{(l)} &= \frac{1}{2}c_{2j}^{(l+1)} - \frac{1}{2}c_{2j+1}^{(l+1)} \quad \text{(differencing)}
\end{align}

Using these formulas, we derive formulas for the coefficients of the inverse Haar wavelet transform.

\textbf{Solution:}
We transform the given forward transform formulas into a system of equations and solve for the finer level coefficients $c_{2j}^{(l+1)}$ and $c_{2j+1}^{(l+1)}$.

\begin{enumerate}
\item \textbf{Add equations (1) and (2):}
\begin{align}
c_j^{(l)} + d_j^{(l)} &= \frac{1}{2}c_{2j}^{(l+1)} + \frac{1}{2}c_{2j+1}^{(l+1)} + \frac{1}{2}c_{2j}^{(l+1)} - \frac{1}{2}c_{2j+1}^{(l+1)}\\
&= c_{2j}^{(l+1)}
\end{align}

\item \textbf{Subtract equation (2) from equation (1):}
\begin{align}
c_j^{(l)} - d_j^{(l)} &= \frac{1}{2}c_{2j}^{(l+1)} + \frac{1}{2}c_{2j+1}^{(l+1)} - \frac{1}{2}c_{2j}^{(l+1)} + \frac{1}{2}c_{2j+1}^{(l+1)}\\
&= c_{2j+1}^{(l+1)}
\end{align}
\end{enumerate}

\textbf{Result:} The inverse Haar wavelet transform formulas are:
\begin{align}
\boxed{c_{2j}^{(l+1)} = c_j^{(l)} + d_j^{(l)}} \quad \text{(even indices)}\\
\boxed{c_{2j+1}^{(l+1)} = c_j^{(l)} - d_j^{(l)}} \quad \text{(odd indices)}
\end{align}

\textbf{Key insight:} The inverse transform reconstructs finer-level coefficients by combining the coarse-level averages with the detail coefficients. Adding recovers even-indexed coefficients, while subtracting recovers odd-indexed coefficients.
\end{example}

\section{Endterm Summer 2024}

\begin{example}[Problem 1a: Quarter-Wave Fourier coefficients are purely imaginary]
Given a real-valued input dataset $f_0, \ldots, f_{2N-1}$ with the symmetry condition $f_{2N-n-1} = -f_n$, show that the Quarter-Wave Fourier coefficients
$$F_k = \frac{1}{2N} \sum_{n=0}^{2N-1} f_n \omega_{2N}^{-k(n+\frac{1}{2})}$$
have only imaginary values and can be written as
$$F_k = -\frac{i}{N} \sum_{n=0}^{N-1} f_n \sin\left(\frac{\pi k}{N}\left(n + \frac{1}{2}\right)\right)$$

\textbf{Solution:}

\begin{enumerate}
\item \textbf{Split the sum:} Separate the sum into two parts
$$F_k = \frac{1}{2N}\left(\sum_{n=0}^{N-1} f_n \omega_{2N}^{-k(n+\frac{1}{2})} + \sum_{n=N}^{2N-1} f_n \omega_{2N}^{-k(n+\frac{1}{2})}\right)$$

\item \textbf{Apply symmetry condition:} Use $f_n = -f_{2N-n-1}$ in the second sum and change variables

\item \textbf{Combine terms:} After simplification, obtain
$$F_k = \frac{1}{2N} \sum_{n=0}^{N-1} f_n \left(\omega_{2N}^{-k(n+\frac{1}{2})} - \omega_{2N}^{k(n+\frac{1}{2})}\right)$$

\item \textbf{Use Euler's formula:} Since $e^{-ix} - e^{ix} = -2i\sin(x)$, we get
$$F_k = -\frac{i}{N} \sum_{n=0}^{N-1} f_n \sin\left(\frac{\pi k}{N}\left(n + \frac{1}{2}\right)\right)$$
\end{enumerate}

\textbf{Key insight:} The anti-symmetry condition $f_{2N-n-1} = -f_n$ causes the real parts to cancel, leaving only imaginary sine components.
\end{example}

\begin{warning}
    Common mistake: Don't forget to check the domain when finding extrema!
\end{warning}

% Section 5: Problem Solving Strategies
\section{Problem Solving Strategies}

\subsection{General Approach}
\begin{enumerate}
    \item \textbf{Understand the problem:} Read carefully and identify what's given and what's asked
    \item \textbf{Plan:} Choose appropriate methods/theorems
    \item \textbf{Execute:} Carry out the calculations
    \item \textbf{Check:} Verify your answer makes sense
\end{enumerate}

\subsection{Specific Techniques}
\begin{reviewbox}[Technique 1: Integration by Parts]
    When to use: Products of functions where one becomes simpler when differentiated
    
    Formula: $\int u\,dv = uv - \int v\,du$
    
    Example types:
    \begin{itemize}
        \item $\int x e^x\,dx$
        \item $\int x \sin x\,dx$
        \item $\int \ln x\,dx$
    \end{itemize}
\end{reviewbox}

% Section 6: Quick Reference
\section{Quick Reference Sheet}

\subsection{Formulas at a Glance}
\begin{center}
\begin{tabular}{|l|l|}
\hline
\textbf{Concept} & \textbf{Formula} \\
\hline
Derivative of $x^n$ & $nx^{n-1}$ \\
\hline
Chain Rule & $(f \circ g)'(x) = f'(g(x)) \cdot g'(x)$ \\
\hline
Product Rule & $(fg)' = f'g + fg'$ \\
\hline
Quotient Rule & $\left(\frac{f}{g}\right)' = \frac{f'g - fg'}{g^2}$ \\
\hline
\end{tabular}
\end{center}

\subsection{Common Pitfalls}
\begin{itemize}
    \item[\important{}] Forgetting the chain rule
    \item[\important{}] Sign errors in integration by parts
    \item[\important{}] Not checking endpoints for absolute extrema
\end{itemize}

% Section 7: Practice Problems
\section{Practice Problems}

\subsection{Basic Problems}
\begin{enumerate}
    \item Find the derivative of $f(x) = x^3 \sin(2x)$
    \item Evaluate $\int_0^{\pi} x \cos x\,dx$
    \item Find all critical points of $g(x) = x^3 - 3x^2 + 2$
\end{enumerate}

\subsection{Advanced Problems}
\begin{enumerate}
    \item Prove that if $f$ is differentiable and $f'(x) > 0$ for all $x \in (a,b)$, 
          then $f$ is strictly increasing on $(a,b)$.
    \item Find the volume of the solid of revolution...
\end{enumerate}

% Section 8: Summary
\section{Summary and Key Takeaways}

\begin{keypoint}
    \textbf{Basis Functions in Sparse Grids}
    \begin{enumerate}
        \item \textbf{Start with 1D hat functions:} The piecewise linear hat function at level $l$ and position $i$ is:
        $$\phi_{l,i}(x) = \max(0, 1 - |2^l \cdot x - i|)$$
        This creates a tent shape with peak value 1 at $x = i \cdot 2^{-l}$, linearly decreasing to 0 at neighboring grid points.
        
        \item \textbf{2D basis functions via tensor product:} 
        $$\phi_{\vec{l},\vec{i}}(\vec{x}) = \phi_{l_1,i_1}(x_1) \cdot \phi_{l_2,i_2}(x_2)$$
        
        \item \textbf{Hierarchical construction:} Each level adds new information (hierarchical surplus) not captured by coarser levels:
        $$W_l = V_l - V_{l-1}$$
        where $V_l$ is the space of piecewise linear functions on grid level $l$.
        
        \item \textbf{Basis functions represent differences:} Each basis function captures what's new at its level, not redundant information from coarser grids.
        
        \item \textbf{Why odd indices only:} Even indices coincide with points from coarser levels. For example, at level 2, index $i=2$ gives point $2 \cdot 2^{-2} = 0.5$, which already exists at level 1 with $i=1$.
    \end{enumerate}
\end{keypoint}

\subsection{Connections to Other Topics}
\begin{itemize}
    \item This material is fundamental for [next topic]
    \item It builds upon [previous topic]
    \item Applications include [real-world example]
\end{itemize}

% Appendices
\appendix
\section{Additional Resources}
\begin{itemize}
    \item Textbook: Chapter X, Sections Y-Z
    \item Online resource: \url{https://example.com}
    \item Practice problems: Problem set 
\end{itemize}

\section{Detailed Proofs}
% Include detailed proofs that were omitted in the main text

\end{document}
